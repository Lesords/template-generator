\documentclass[11pt]{article}		% 设置字体大小

\usepackage{lmodern}
\usepackage{amssymb,amsmath}
\usepackage{ifxetex,ifluatex}
\usepackage{fixltx2e} % provides \textsubscript
\ifnum 0\ifxetex 1\fi\ifluatex 1\fi=0 % if pdftex
  \usepackage[T1]{fontenc}
  \usepackage[utf8]{inputenc}
\else % if luatex or xelatex
  \ifxetex
    \usepackage{mathspec}
  \else
    \usepackage{fontspec}
  \fi
  \defaultfontfeatures{Ligatures=TeX,Scale=MatchLowercase}
\fi
% use upquote if available, for straight quotes in verbatim environments
\IfFileExists{upquote.sty}{\usepackage{upquote}}{}
% use microtype if available
\IfFileExists{microtype.sty}{%
\usepackage[]{microtype}
\UseMicrotypeSet[protrusion]{basicmath} % disable protrusion for tt fonts
}{}
\PassOptionsToPackage{hyphens}{url} % url is loaded by hyperref
\usepackage[unicode=true]{hyperref}
\hypersetup{
            pdfauthor={Lese},
            pdfborder={0 0 0},
            breaklinks=true}
\urlstyle{same}  % don't use monospace font for urls
\usepackage[top=20mm,left=20mm,right=20mm,bottom=20mm]{geometry}
\IfFileExists{parskip.sty}{%
\usepackage{parskip}
}{% else
\setlength{\parindent}{0pt}
\setlength{\parskip}{6pt plus 2pt minus 1pt}
}
\setlength{\emergencystretch}{3em}  % prevent overfull lines
\providecommand{\tightlist}{%
  \setlength{\itemsep}{0pt}\setlength{\parskip}{0pt}}
\setcounter{secnumdepth}{0}
% Redefines (sub)paragraphs to behave more like sections
\ifx\paragraph\undefined\else
\let\oldparagraph\paragraph
\renewcommand{\paragraph}[1]{\oldparagraph{#1}\mbox{}}
\fi
\ifx\subparagraph\undefined\else
\let\oldsubparagraph\subparagraph
\renewcommand{\subparagraph}[1]{\oldsubparagraph{#1}\mbox{}}
\fi

% set default figure placement to htbp
\makeatletter
\def\fps@figure{htbp}
\makeatother

\usepackage{xeCJK}
\usepackage{minted}
\usepackage{graphicx}
\usepackage{fancyhdr}
\setminted{breaklines}
\setminted{linenos}


\date{}

\begin{document}

\begin{titlepage}		% 封面设置
\maketitle

\vspace{2cm}
\centerline{\title{\fontsize{50pt}{15pt} \textbf{ACM算法模板}}}


\vspace{5cm}
\centerline{\includegraphics[width=12cm]{icpcfoundation.jpg}}

\vspace{1.6cm}
\centerline{\author{\Huge Lese}}

\vspace{0.5cm}
\centerline{\date{\huge \today}}

\thispagestyle{empty}
\end{titlepage}


\clearpage				% 空白页
\thispagestyle{empty}	% 无页码
\phantom{s}
\clearpage

\newpage
\setcounter{page}{1} 	% 从当前位置开始设置页码

\pagestyle{fancy} 		% 目录页设置页眉
\fancyhead[RO,LE]{\textbf{ACM 模板  by Lese}}

{
\setcounter{tocdepth}{3}
\tableofcontents
}


\newpage

\pagestyle{fancy}

\fancyhead[RO,LE]{\textbf{ACM 模板  by Lese}}

\section{1 字符串}\label{ux5b57ux7b26ux4e32}

\subsection{1.1 字符串哈希}\label{ux5b57ux7b26ux4e32ux54c8ux5e0c}

\begin{minted}[]{cpp}
//const int M = 1e9 + 7;//小质数
const ll MOD = 212370440130137957ll;//大质数
int idx(char ch) {//得到字符对应的数值
    if(ch>='0'&&ch<='9') return ch-'0';
    if(ch>='a'&&ch<='z') return ch-'a'+10;
    if(ch>='A'&&ch<='Z') return ch-'A'+36;
}
//s为对应字符串,bas为对应进制数
ull get_hash(const string& s, const int bas) {//131或13331
    ull ans = 0;
    int len = s.size();
    for(int i = 0;i < len;i++) {
        ans = (ans*bas+idx(s[i]))%MOD;
    }
    return ans;
}
\end{minted}

\subsection{1.2 KMP算法}\label{kmpux7b97ux6cd5}

\begin{minted}[]{cpp}
const int MAXN = 1e6+6;
int len_n,len_m,nextt[MAXN];//注意先处理n和m的长度
char n[MAXN],m[MAXN];//n为长串,m为短串
void get_nextt() {//自身匹配
    int i = 0,j = -1;
    nextt[i] = j;
    while(i<len_m) {//目标串长度
        if(m[i] == m[j]||j == -1) nextt[++i] = ++j;
        else j = nextt[j];//自身回退
    }
}
bool kmp() {
    int i = 0, j = 0;
    get_nextt();
    while(i<len_n&&j<len_m) {
        if(n[i]==m[j]||j==-1) i++,j++;//匹配成功都会后移!!!
        else j = nextt[j];//回退
    }
    if(j == len_m) return true;//len_m表示匹配成功
    return false;
}
\end{minted}

\subsection{1.3 EXKMP算法}\label{exkmpux7b97ux6cd5}

\begin{minted}[]{cpp}
//nxt[i]: b[i···n-1]与b的LCP值
//extend[i]: a[i···n-1]与b的LCP值
const int MAXN = 2e7+5;
int nxt[MAXN], extend[MAXN];
void pre_exkmp(string s) {
    int n = (int)s.length();
    nxt[0] = n;
    //[l, r]为z-box区间
    for(int i = 1, l = 0, r = 0;i < n;++i) {
        if(i <= r && nxt[i - l] < r - i + 1) {//r-i+1为剩余区间大小
            nxt[i] = nxt[i - l];
        }
        else {
            nxt[i] = max(0, r - i + 1);
            while(i + nxt[i] < n && s[nxt[i]] == s[i + nxt[i]]) ++nxt[i];
        }
        if(i + nxt[i] - 1 > r) l = i, r = i + nxt[i] - 1;
    }
}
void exkmp(string a, string b) {//a为长串,b为短串
    pre_exkmp(b);
    int n = a.size();
    while(a[extend[0]] == b[extend[0]]) extend[0]++;//特判第一位置的值
    for(int i = 1,l = 0, r = 0;i < n; ++i) {
        if(i <= r && nxt[i-l] < r - i + 1) {
            extend[i] = nxt[i - l];
        }
        else {
            extend[i] = max(0, r - i + 1);
            while(i + extend[i] < n && b[extend[i]] == a[i + extend[i]]) 
                ++extend[i];
        }
        if(i + extend[i] - 1 > r) l = i, r = i + extend[i] - 1;
    }
}
\end{minted}

\subsection{1. 4 Manacher算法}\label{manacherux7b97ux6cd5}

\begin{minted}[]{cpp}
const int MAXN = 11e6 + 5;
char s[MAXN], ma[MAXN*2];
int mp[MAXN*2];//num[i] 为在s中以对应位置为中心的极大子回文串的总长度+1
int Manacher(char *s, int siz){//siz为字符串长度
    int len = 0;
    ma[len++] = '$'; ma[len++] = '#';
    for(int i = 0;i < siz;i++){
        ma[len++] = s[i];
        ma[len++] = '#';
    }
    ma[len] = 0;//末尾为空字符
    int mx = 0, id = 0;//id为局部最大回文串中心点,mx为右边界
    int res = 0;
    for(int i = 0;i < len;i++) {
        //由于对于i的对称点j(j小)来说:id - j = i - id 推出 j = 2id-i
        mp[i] = mx > i ? min(mp[2*id-i], mx - i) : 1;
        while(ma[i+mp[i]] == ma[i-mp[i]]) mp[i]++;//暴力枚举
        if(i + mp[i] > mx) {//更新局部最大回文串
            mx = mp[i] + i;
            id = i;
        }
        res = max(res, mp[i] - 1);
        //mp[i]为左部分大小,但是包括了#的数量,而且#数量比原字符串多,所以需-1
    }
    return res;//为最长回文串长度
}
\end{minted}

\subsection{1.5 AC自动机}\label{acux81eaux52a8ux673a}

\begin{minted}[]{cpp}
const int MAXN = 1e6+5;
char s[MAXN];
namespace AC {
    int tr[MAXN][26], tot;
    int e[MAXN], fail[MAXN];//e记录节点对应的字符串出现次数
    queue<int> q;
    void init() {
        for(int i = 0;i <= tot;i++) {//把用过的节点清空
            memset(tr[i], 0, sizeof tr[i]);
            fail[i] = e[i] = 0;
        }
        tot = 0;
    }
    void insert(char *s) {//插入一个字符串(建字典树)
        int u = 0;
        int len = strlen(s + 1);
        for(int i = 1;i <= len;i++) {//字符串从1开始记录
            if(!tr[u][s[i] - 'a']) {
                tr[u][s[i] - 'a'] = ++tot;
            }
            u = tr[u][s[i] - 'a'];
        }
        e[u] ++;
    }
    void build() {//建fail树
        for(int i = 0;i < 26;i++) if(tr[0][i]) q.push(tr[0][i]);
        while(q.size()) {
            int u = q.front();q.pop();
            for(int i = 0;i < 26;i++) {
                if(tr[u][i]) fail[tr[u][i]] = tr[fail[u]][i],q.push(tr[u][i]);
                else tr[u][i] = tr[fail[u]][i];
            }
        }
    }
    int query(char *t) {//t为需要查询的字符串
        int u = 0, res = 0;
        for(int i = 1;t[i];i++) {//字符串从位置1开始匹配
            u = tr[u][t[i] - 'a'];
            for(int j = u;j && e[j] != -1;j = fail[j])
                res += e[j], e[j] = -1;
        }
        return res;
    }
}
int main() {
    AC::init();//清空数组
    int n;cin>>n;
    for(int i = 1;i <= n;i++) {
        cin>>(s+1);
        AC::insert(s);
    }
    AC::build();//记得建fail树!!!
    cin>>(s+1);
    cout<<AC::query(s);
    return 0;
}
\end{minted}

\subsection{1.6 字典树}\label{ux5b57ux5178ux6811}

\begin{minted}[]{cpp}
struct Trie {
    static const int SIZE = 2e5+5;
    int nex[SIZE][26], cnt;
//    bool exist[SIZE];  // 该结点结尾的字符串是否存在
    void init() {
        cnt = 0;//0为根节点,无对应值
        memset(nex, 0, sizeof(nex));
//        memset(exist, 0, sizeof(exist));
    }
    int insert(char *s, int l) {  // 插入字符串,l为字符串长度
        int p = 0;
        for (int i = 0; i < l; i++) {
            int c = s[i] - 'a';
            if (!nex[p][c]) nex[p][c] = ++cnt;  // 如果没有,就添加结点
            p = nex[p][c];
        }
//        exist[p] = 1;//类型为void时
        return p;//返回对应节点
    }
    int find(char *s, int l) {  // 查找字符串,l为字符串长度
        int p = 0;
        for (int i = 0; i < l; i++) {
            int c = s[i] - 'a';
            if (!nex[p][c]) return 0;
            p = nex[p][c];
        }
        return p;//返回对应节点(0为不存在
//        return exist[p];//用bool类型时使用
    }
}t;
\end{minted}

\newpage

\pagestyle{fancy}

\fancyhead[RO,LE]{\textbf{ACM 模板  by Lese}}

\section{2 数论}\label{ux6570ux8bba}

\subsection{2.1 埃式筛}\label{ux57c3ux5f0fux7b5b}

\begin{minted}[]{cpp}
const int MAXV = 1e6;
const int MAXN = 8e4+8;//共78498个质数
int prim[MAXN], v[MAXV+10], cnt;//v判断当前数是否为质数
void get_prim() {
    for(int i = 2;i <= MAXV;i++) {
        if(v[i]) continue;
        prim[++cnt] = i;
        for(int j = i+i;j <= MAXV;j+=i) {
            v[j] = 1;
        }
    }
}
\end{minted}

\subsection{2.2 线性筛}\label{ux7ebfux6027ux7b5b}

\subsubsection{2.2.1 质数}\label{ux8d28ux6570}

\begin{minted}[]{cpp}
const int MAXV = 1e8;
const int MAXN = 6e6+6;//共5761455个质数
int prim[MAXN], v[MAXV+10], cnt;//v记录最小质因子
void get_prim() {
    for(int i = 2;i <= MAXV;i++) {
        if(v[i]==0) {
            prim[++cnt] = i;
            v[i] = i;
        }
        for(int j = 1;j <= cnt;j++) {
            if(i*1ll*prim[j]>MAXV||v[i]<prim[j]) break;
            v[i*prim[j]] = prim[j];
        }
    }
}
\end{minted}

\subsubsection{2.2.2 欧拉函数}\label{ux6b27ux62c9ux51fdux6570}

\(1 \sim N\) 中与 \(N\) 互质的数的个数被称为欧拉函数,记为
\(\varphi(N)\) 。

\begin{minted}[]{cpp}
const int MAXN = 1e6+5;
int v[MAXN], prim[MAXN], phi[MAXN], cnt;//phi为欧拉函数值,v为最小质因子
void euler(int n) {
    phi[1] = 1;
    for(int i = 2;i <= n;i++) {
        if(v[i]==0) {
            v[i] = i, prim[++cnt] = i;
            phi[i] = i-1;
        }
        for(int j = 1;j <= cnt;j++) {
            if(v[i]<prim[j] || prim[j]>n/i) break;
            v[i*prim[j]] = prim[j];
            phi[i*prim[j]] = phi[i]*( i%prim[j] ?prim[j]-1:prim[j]);
        }
    }
}
\end{minted}

\subsubsection{2.2.3
莫比乌斯函数}\label{ux83abux6bd4ux4e4cux65afux51fdux6570}

当N包含\textbf{相等的质因子}时,\(\mu (N) = 0\)。当N的所有质因子各不相等时,若N有\textbf{偶数个质因子},\(\mu(N) = 1\),若N有\textbf{奇数个质因子},\(\mu (N) = -1\)。

\begin{minted}[]{cpp}
const int MAXN = 1000000;
bool check[MAXN+10];
int prime[MAXN+10], mu[MAXN+10];
void Moblus(){
    memset(check,false,sizeof(check));
    mu[1] = 1;
    int tot = 0;
    for(int i = 2; i <= MAXN; i++) {
        if(!check[i]) {
            prime[tot++] = i;
            mu[i] = -1;
        }
        for(int j = 0; j < tot; j++) {
            if(i * prime[j] > MAXN) break;
            check[i * prime[j]] = true;
            if( i % prime[j] == 0){
                mu[i * prime[j]] = 0;
                break;
            }
            else{
                mu[i * prime[j]] = -mu[i];
            }
        }
    }
}
\end{minted}

\subsection{2.3 Lucas定理}\label{lucasux5b9aux7406}

\begin{minted}[]{cpp}
ll f[MAXN];//存储阶乘
void init(int p) {//记得先初始化阶乘!!!
    f[0] = 1;
    for(int i = 1;i < p;i++) {
        f[i] = (f[i-1] * i)%p;
    }
}
ll poww(ll a,ll b, ll p) {//a^b
    ll ans = 1;
    while(b) {
        if(b&1) ans = (ans*a)%p;
        a = (a*a)%p;
        b >>= 1;
    }
    return ans%p;
}
ll C(ll n, ll m, ll p) {//C(n, m)
    if(m>n) return 0;
    return ((f[n]*poww(f[m], p-2, p))%p * poww(f[n-m], p-2, p))%p;
}
ll lucas(ll n, ll m, ll p) {
    if(m==0) return 1;
    return (lucas(n/p, m/p, p) * C(n%p, m%p, p)) %p;
}
\end{minted}

\subsection{2.4 求逆元}\label{ux6c42ux9006ux5143}

\subsubsection{2.4.1 费马小定理}\label{ux8d39ux9a6cux5c0fux5b9aux7406}

当模数 \(p\) 为\textbf{质数}时, \(b^{p-2}\) 即为 \(b\) 的乘法逆元

\begin{minted}[]{cpp}
ll poww(ll a, ll b, ll p) {//a的b次方
    ll ans = 1;//ans为结果
    while(b) {
        if(b&1) ans = (ans*a)%p;
        a = (a*a)%p;//二进制每次乘都要翻倍 
        b >>= 1;
    }
    return ans%p;   
}
ll inv(ll a, ll p) { return pow(a, p - 2, p); }
\end{minted}

\subsubsection{2.4.2
扩展欧几里得算法}\label{ux6269ux5c55ux6b27ux51e0ux91ccux5f97ux7b97ux6cd5}

\begin{minted}[]{cpp}
ll exgcd(ll a, ll b, ll &x,ll &y) {
    if(b==0) { x = 1, y = 0; return a;}
    ll d = exgcd(b, a%b, x, y);
    ll z = x; x = y; y = z - y * (a/b);
    return d;
}
int main() {
    ll a, b, x, y;
    cin>>a>>b;
    exgcd(a, b, x, y);
    cout<<(x%b+b)%b;//由于原式中mod的值是b,所以可以%b(输出即为逆元的值)
    return 0;
}
\end{minted}

\subsubsection{2.4.3 线性求法}\label{ux7ebfux6027ux6c42ux6cd5}

\begin{minted}[]{cpp}
inv[1] = 1;//inv为longlong数组
for(int i = 2; i < p; ++ i)
    inv[i] = (p - p / i) * inv[p % i] % p;//取模p所以可以 +p
\end{minted}

\subsection{2.5 矩阵快速幂}\label{ux77e9ux9635ux5febux901fux5e42}

\begin{minted}[]{cpp}
const int MAXN = 105;
const int MOD = 1e9+7;
struct matrix{
    int r, c;
    ll s[MAXN][MAXN];
    matrix(int r = 0,int c = 0):r(r),c(c) {
        memset(s, 0, sizeof s);
    }
};
matrix operator* (const matrix& a, const matrix& b) {
    matrix c = matrix(a.r, b.c);
    for (int i = 0; i < c.r;i++) {
        for (int j = 0; j < c.c;j++)
            for (int k = 0; k < a.c;k++)
                // c.s[i][j] += a.s[i][k] * b.s[k][j];
                c.s[i][j] = (c.s[i][j] + a.s[i][k]*b.s[k][j]%MOD)%MOD;
    }
    return c;
}
matrix power(matrix a,ll b) {
    matrix res = a;
    b--;
    while(b) {
        if(b&1) res = res * a;
        a = a*a;
        b >>= 1;
    }
    return res;
}
int main() {
    ll n, k;
    cin>>n>>k;
    matrix a(n, n);
    for(int i = 0;i < n;i++) {
        for(int j = 0;j < n;j++) cin>>a.s[i][j];
    }
    matrix ans = power(a, k);
    for(int i = 0;i < n;i++) {
        for(int j = 0;j < n;j++) {
            if(j) cout<<" ";
            cout<<(ans.s[i][j]+MOD)%MOD;
        }
        cout<<"\n";
    }
    return 0;
}
\end{minted}

\subsection{2.6 O(1)快速乘}\label{o1ux5febux901fux4e58}

\begin{minted}[]{cpp}
long long mul(long long a, long long b, long long p) {
    return (a * b - (long long) (a / (long double) p * b + 1e-3) * p + p) % p;
}
\end{minted}

\subsection{2.7 Miller-Rabin}\label{miller-rabin}

快速判断一个数是否为质数

\begin{minted}[]{cpp}
long long poww(long long a, long long n, long long mod) {
    long long res = 1;//mul为快速乘(3.6)
    for (; n; n >>= 1, a = mul(a, a, mod)) if (n & 1) res = mul(res, a, mod);
    return res;
}
bool isPrime(long long n) {
    const static int primes[12] = {2, 3, 5, 7, 11, 13, 17, 19, 23, 29, 31, 37};

    long long s = 0, d = n - 1;
    while (d % 2 == 0) d /= 2, s++;
    if (s == 0) return n == 2;

    for (int i = 0; i < 12 && primes[i] < n; i++) {
        long long a = primes[i];
        if (poww(a, d, n) != 1) {
            bool flag = true;
            for (int r = 0; r < s; r++)
                if (flag && poww(a, d * (1 << r), n) == n - 1) flag = false;
            if (flag) return false;
        }
    }

    return true;//true为是质数
}
\end{minted}

\subsection{2.8 Pollard's Rho}\label{pollards-rho}

快速分解质因子

\begin{minted}[]{cpp}
long long gcd(long long a, long long b) {
    return b ? gcd(b, a % b) : a;
}
namespace PollardRho {
    long long g(long long x, long long n, long long c) {
        return (mul(x, x, n) + c) % n;//mul为快速乘(3.6)
    }
    long long rho(long long n, long long c) {
        long long x = rand() % n, y = x, d = 1;

        for (long long i = 1, k = 2; d == 1; i++) {
            x = g(x, n, c);
            d = gcd(x > y ? x - y : y - x, n);
            if (x == y) return n;
            if (i == k) k <<= 1, y = x;
        }

        return d;
    }
    void find(long long n, long long c, std::map<long long, int> &res) {
        if (n == 1) return;
        if (isPrime(n)) {//isPrime为Miller-Rabin(3.7)
            res[n]++;
            return;
        }

        long long p = n;
        while (p == n) p = rho(p, c++);
        find(p, c, res);
        find(n / p, c, res);
    }
    std::map<long long, int> divide(long long n) {
        static std::map<long long, int> res;
        res.clear();
        find(n, 1, res);
        return res;//res.begin()->fi即为最小质因子
    }
}
\end{minted}

\subsection{2.9 BGSG算法}\label{bgsgux7b97ux6cd5}

求解式子 \(a^x \equiv b (mod \ n)\) ,其中 \(0 <= x < n\)。

\textbf{注意:需要满足a和n互质,但是n是素数和不是素数都可以(不限条件)}

\begin{minted}[]{cpp}
#define MOD 76543
ll hs[MOD],head[MOD],nextt[MOD],id[MOD],top;
void insert(int x,int y){
    ll k = x%MOD;
    hs[top] = x, id[top] = y, nextt[top] = head[k], head[k] = top++;
}
ll find(ll x){
    ll k = x%MOD;
    for(int i = head[k]; i != -1; i = nextt[i])
        if(hs[i] == x)
        return id[i];
    return -1;
}
ll BSGS(ll a, ll b, ll n) {//a为底数,b为值,n为模数
    memset(head, -1, sizeof(head));
    top = 1;
    if(b == 1)return 0;
    ll m = sqrt(n*1.0), j;
    long long x = 1, p = 1;
    for(int i = 0; i < m; ++i, p = p*a%n)insert(p*b%n,i);
    for(long long i = m; ;i += m){
        if( (j = find(x = x*p%n)) != -1) return i-j;
        if(i > n)break;
    }
    return -1;
}
\end{minted}

\subsection{2.10 斯特林数}\label{ux65afux7279ux6797ux6570}

\begin{enumerate}
\def\labelenumi{\arabic{enumi}.}
\tightlist
\item
  把n个不同的球分成r个非空循环排列的方法数
\end{enumerate}

递推形式 : \$s(n, r) = (n-1)s(n-1, r) + s(n-1, r-1), n \textgreater{} r
≥ 1 \$

\begin{enumerate}
\def\labelenumi{\arabic{enumi}.}
\setcounter{enumi}{1}
\tightlist
\item
  把n个不同的球放到r个盒子里,假设没有空盒,则放球方案数记作\(S(n,r)\),称为第二类Stirling数
\end{enumerate}

\(S(n,r)=rS(n−1,r)+S(n−1,r−1),n>r≥1\) \newpage
 \pagestyle{fancy}

\fancyhead[RO,LE]{\textbf{ACM 模板  by Lese}}

\section{3 数据结构}\label{ux6570ux636eux7ed3ux6784}

\subsection{3.1 RMQ}\label{rmq}

\begin{minted}[]{cpp}
const int MAXN = 1e5+5;
int f[MAXN][20];
void RMQ_init() {
    for(int i = 1;i <= n;i++) f[i][0] = a[i];//边界
    int siz = log(n)/log(2);//siz只用于计算倍增规模!!!
    for(int i = 1;i <= siz;i++) {
        //j为起点, j+(1<<i)-1为终点
        //i大区间分成两个(i-1)小区间
        for(int j = 1;j+(1<<i)-1 <= n;j++) {
            int len = 1<<(i-1);//大区间分成两个小区间
            f[j][i] = max(f[j][i-1], f[j+len][i-1]);
        }
    }
}
int query(int l,int r) {// 求[l,r]区间最大值
    int siz = log(r-l+1)/log(2);
    return max(f[l][siz],f[r-(1<<siz)+1][siz]);
}
\end{minted}

\subsection{3.2
树状数组求第k大值}\label{ux6811ux72b6ux6570ux7ec4ux6c42ux7b2ckux5927ux503c}

树状数组记录的是对应数字出现的次数

\begin{minted}[]{cpp}
int kth(int rk) {//求第rk大
    int siz = log2(n);//最大的幂
    int num = 0,pos = 0;//num为当前统计数量,pos为当前计算的位置
    rk = cnt-rk+1;//转换为第 cnt-rk+1 小
    for(int i = siz;~i;i--) {//计算到-1时结束
        pos += 1<<i;//累加位置
        //位置不能大于等于n,并且数量不能大于等于rk
        //也就是说到达一个边界,使得当前位置都不满足,那么下面一个位置一定是满足条件的
        if(pos>=n||num + tree[pos]>=rk) {
            pos -= 1<<i;
        }
        else num += tree[pos];//满足条件,累加数量
    }
    return pos+1;//下一个位置即为结果
}
\end{minted}

\subsection{3.3 主席树}\label{ux4e3bux5e2dux6811}

查询区间\([l, r]\)中第k小的值

\begin{minted}[]{cpp}
const int N = 200000 + 5;
struct Seg {
    int ls, rs;
    int sum;
}t[N*40];
int rt[N], tot;
vector<int> all;
int n, a[N], l, r, k, m;
int build(int l, int r) {
    int p = ++tot;
    if(l == r) {
        t[p].sum = 0;
        return p;
    }
    int mid = (l + r) >> 1;
    t[p].ls = build(l, mid);
    t[p].rs = build(mid+1, r);
    t[p].sum = 0;
    return p;
}
int insert(int now, int l, int r, int pos, int val) {
    int p = ++tot;
    t[p] = t[now];
    if(l == r){
        t[p].sum += val;
        return p;
    }
    int mid = (l + r) >> 1;
    if(pos <= mid) t[p].ls = insert(t[now].ls, l, mid, pos, val);
    else t[p].rs = insert(t[now].rs, mid+1, r, pos, val);
    t[p].sum = t[t[p].ls].sum + t[t[p].rs].sum;
    return p;
}
int ask(int p, int q, int l, int r, int k) {
    if(l == r) return l;
    int mid = (l + r) >> 1;
    int lcnt = t[t[q].ls].sum - t[t[p].ls].sum;
    if(k <= lcnt) return ask(t[p].ls, t[q].ls, l, mid, k);
    else return ask(t[p].rs, t[q].rs, mid+1, r, k - lcnt);
}
int main(){
    scanf("%d%d", &n, &m);
    for(int i = 1;i <= n;i++) {
        scanf("%d", &a[i]);
        all.push_back(a[i]);
    }
    sort(all.begin(), all.end());
    all.erase(unique(all.begin(), all.end()), all.end()); //离散化
    for(int i = 1;i <= n;i++) {
        a[i] = lower_bound(all.begin(), all.end(), a[i]) - all.begin() + 1;
    }
    rt[0] = build(1, all.size());//初始线段树
    for(int i = 1;i <= n;i++) {
        rt[i] = insert(rt[i-1], 1, all.size(), a[i], 1);
    }
    while(m--) {
        scanf("%d%d%d", &l, &r, &k);//[l, r] 区间内第k大值
        printf("%d\n", all[ask(rt[l-1], rt[r], 1, all.size(), k) - 1]);
    }
    return 0;
}
\end{minted}

\newpage

\pagestyle{fancy}

\fancyhead[RO,LE]{\textbf{ACM 模板  by Lese}}

\section{4 图论}\label{ux56feux8bba}

\subsection{4.1 最短路}\label{ux6700ux77edux8def}

\subsubsection{4.1.1 Dijkstra
单源最短路径}\label{dijkstra-ux5355ux6e90ux6700ux77edux8defux5f84}

\begin{minted}[]{cpp}
#include<iostream>
#include<cstring>
#include<algorithm>
using namespace std;
const int INF = 0x3f3f3f3f;
const int MAXN = 105;
int map[MAXN][MAXN],book[MAXN],dis[MAXN],n;
void dijkstra(int s) {
    for(int i = 1;i <= n;i++) {//初始化dis数组
        dis[i]=map[s][i];
    }
    memset(book, 0, sizeof(book));//初始化
    book[s] = 1;//标记
    for(int i = 1;i < n;i++) {
        int minn = INF,u;
        for(int j = 1;j <= n;j++) {
            if(!book[j] && dis[j]<minn) {//寻找最短路
                minn = dis[j];
                u = j;
            }
        }
        book[u] = 1;//标记该点
        for(int j = 1;j <= n;j++) {
            if(map[u][j]<INF)
                dis[j] = min(dis[j],dis[u]+map[u][j]);//更新
        }
    }
}
int main() {
    int s,u,v,w,m;
    cin>>n>>m;
    for(int i = 1;i <= n;i++)//初始化
        for(int j = i;j <= n;j++)
        map[i][j] = map[j][i] = (i==j)?0:INF;
    for(int i = 1;i <= m;i++) {
        cin>>u>>v>>w;
        map[u][v] = map[v][u] = min(w, map[u][v]);
    }
    cin>>s;//起点
    dijkstra(s);
    for(int i = 1;i <= n;i++) {
        cout<<dis[i]<<' ';
    }
    cout<<endl;
    return 0;
}
\end{minted}

\subsubsection{4.1.2 Dijkstra 算法 +
堆优化}\label{dijkstra-ux7b97ux6cd5-ux5806ux4f18ux5316}

\begin{minted}[]{cpp}
#include<iostream>
#include<cstring>
#include<queue>
#include<vector>
#include<algorithm>
#define ll long long
#define INF 0x3f3f3f3f3f3f3f3f
using namespace std;
const int MAXN = 1e5+5;
struct edge {
    ll dot,val;
    friend bool operator < (edge a,edge b) {
        return a.val > b.val;//先小
    }
};
ll n,m,s,vis[MAXN],dis[MAXN];//vis判断当前是否已经确定,dis为长度
vector<edge> g[MAXN];//存图
void dijkstra(int dot) {
    for(int i = 1;i <= n;i++) dis[i] = INF;         //初始化
    dis[dot] = 0;                                   //源点为0
    priority_queue<edge> q;                         //优先队列
    q.push(edge{dot,0});                            //源点入队列
    while(!q.empty()) {
        edge tmp = q.top();q.pop();
        if(vis[tmp.dot]) continue;                  //跳过已确定的
        else vis[tmp.dot] = 1;                      //否则标记
        for(int i = 0;i < g[tmp.dot].size();i++) {  //遍历所有可到达点
            edge now = g[tmp.dot][i];
            if(dis[now.dot]>dis[tmp.dot]+now.val) { //松弛操作
                dis[now.dot] = dis[tmp.dot]+now.val;
                //未访问,则继续入队
                if(!vis[now.dot]) q.push(edge{now.dot,dis[now.dot]});
            }
        }
    }
}
int main() {
    cin>>n>>m>>s;
    for(int i = 1;i <= m;i++) {
        ll u,v,w;
        cin>>u>>v>>w;
        g[u].push_back(edge{v,w});                  //建图
    }
    dijkstra(s);                                    //跑dijkstra
    for(int i = 1;i <= n;i++) {
        if(i!=1) cout<<" ";
        cout<<dis[i];
    }
    return 0;
}
\end{minted}

\subsubsection{4.1.3 单源最短路
SPFA}\label{ux5355ux6e90ux6700ux77edux8def-spfa}

\begin{minted}[]{cpp}
#include<iostream>
#include<vector>
#include<queue>
#include<algorithm>
#define ll long long
#define INF 0x3f3f3f3f
using namespace std;
const int MAXN = 1e5+5;
struct edge{
    ll dot,val;//dot为指向点,val为边值
};
ll n,m,s,dis[MAXN];         //dis保存距离
bool vis[MAXN];             //判断是否在队列中(true为可入队)
vector<edge> g[MAXN];       //存图
void spfa(int dot) {
    for(int i = 1;i <= n;++i) dis[i] = INF,vis[i] = true;//初始化
    queue<int> q;
    q.push(dot);                                //初始化入队源点
    dis[dot] = 0;                               //源点的距离为0
    vis[dot] = false;                           //不可入队
    while(!q.empty()) {                         //若队列不为空
        int tmp = q.front();q.pop();
        vis[tmp] = true;                        //出入之后修改vis
        for(int i = 0;i < g[tmp].size();++i) {
            edge now = g[tmp][i];               //指向点及边值
            if(dis[now.dot]>dis[tmp]+now.val) { //可松弛
                dis[now.dot] = dis[tmp] + now.val;//取小值
                if(vis[now.dot]) {              //判断是否可以入队
                    q.push(now.dot);
                    vis[now.dot] = false;       //更新vis
                }
            }
        }
    }
}
int main() {
    cin>>n>>m>>s;
    for(int i = 1;i <= m;i++) {
        int u,v,w;
        cin>>u>>v>>w;
        g[u].push_back(edge{v,w});          //建图
    }
    spfa(s);                                //从s点开始spfa
    for(int i = 1;i <= n;i++) {
        if(i!=1) cout<<" ";
        if(dis[i]==INF) cout<<(1<<31)-1;    //不可到的特殊答案
        else cout<<dis[i];
    }
    cout<<"\n";
    return 0;
}

\end{minted}

\subsection{4.2 最小生成树}\label{ux6700ux5c0fux751fux6210ux6811}

\subsubsection{4.2.1 Prim算法}\label{primux7b97ux6cd5}

\begin{minted}[]{cpp}
#define INF 0x3f3f3f3f
const int MAXN = 5e3+5;
int a[MAXN][MAXN], d[MAXN], n, m, ans;                  //a记录原图,d记录当前较短距离
bool v[MAXN];                                           //v记录当前点是否被选过
int prim(int u) {//u为起点
    int ans = 0;
    for(int i = 1;i <= n;i++) {                         //记录所有从u开始的路径长度
        d[i] = a[u][i];
    }
    memset(v, 0, sizeof v);
    v[u] = 1;
    for(int i = 1;i < n;i++) {
        int x = 0;                                      //x为距离较短的点编号
        for(int j = 1;j <= n;j++)
            if(!v[j] && (x == 0 || d[j] < d[x])) x = j;
        if(d[x]==INF) return -1;                        //原图不连通
        v[x] = 1;
        ans += d[x];
        for(int y = 1;y <= n;y++)
            if(!v[y]) d[y] = min(d[y], a[x][y]);        //更新对应距离
    }
    return ans;
}
int main() {
    cin >> n >> m;
    memset(a, 0x3f, sizeof a);
    for(int i = 1;i <= n;i++) a[i][i] = 0;              //自环距离为0
    for(int i = 1;i <= m;i++) {
        int x,y,z;
        scanf("%d%d%d", &x, &y, &z);
        a[y][x] = a[x][y] = min(a[x][y], z);            //更新为较小值
    }   
    int ans = prim(1);
    if(ans==-1) cout<<"orz";                            //不连通
    else cout<<ans;
    return 0;
}
\end{minted}

\subsubsection{4.2.2 Prim+堆优化}\label{primux5806ux4f18ux5316}

\begin{minted}[]{cpp}
const int MAXN = 5e3+5;
struct node {
    int dot, val;
    friend bool operator < (node a, node b) {
        return a.val > b.val;                       //优先权值小的
    }
};
int n, m;
bool v[MAXN];                                       //v记录当前点是否被选过
vector<node> g[MAXN];
int prim(int u) {                                   //u为起点
    priority_queue<node> q;
    memset(v, 0, sizeof v);                         //初始化所有点未访问
    v[u] = 1;                                       //标记起点
    int ans = 0;
    for(int i = 0;i < g[u].size();i++) {            //起点可以到达的所有位置
        q.push(g[u][i]);
    }
    for(int i = 1;i < n;i++) {
        while(!q.empty()&&v[q.top().dot]) q.pop();  //先清空所有已访问的点
        if(q.empty()) return -1;                    //若为空,表示原图不连通
        node now = q.top();q.pop();                 //取队首的点
        v[now.dot] = 1;
        ans += now.val;
        for(int j = 0;j < g[now.dot].size();j++) {  //遍历该点的所有指向点
            node tmp = g[now.dot][j];
            if(v[tmp.dot]) continue;                //跳过已访问的点
            q.push(tmp);
        }
    }
    return ans;
}
int main() {
    cin >> n >> m;
    for(int i = 1;i <= m;i++) {
        int x, y, z;
        scanf("%d%d%d", &x, &y, &z);
        g[x].push_back({y, z});                         //存边
        g[y].push_back({x, z});
    }
    int ans = prim(1);
    if(ans==-1) cout<<"orz";                            //不连通
    else cout<<ans;
    return 0;
}
\end{minted}

\subsubsection{4.2.3 Kruskal算法}\label{kruskalux7b97ux6cd5}

\begin{minted}[]{cpp}
#include<iostream>//ac
#include<vector>
#include<algorithm>
using namespace std;
const int MAXN = 2e5+5;
struct node{
    int x,y,z;
    friend bool operator < (node a,node b) {
        return a.z < b.z;                       //优先边权小的
    }
}a[MAXN];
int dis[MAXN];                                  //记录祖先节点
int getf(int p) {                               //求祖先,路径压缩
    return dis[p]==p?p:dis[p]=getf(dis[p]);
}
int main() {
    int n,m;
    cin>>n>>m;
    for(int i = 1;i <= n;i++) dis[i] = i;       //初始化祖先为本身
    for(int i = 1;i <= m;i++) {
        int x,y,z;
        cin>>x>>y>>z;
        a[i] = {x,y,z};                         //存边
    }
    sort(a+1,a+1+m);                            //按边权从小到大排序
    int ans = 0, cnt = 1;                       //初始化连通块数量为1
    for(int i = 1;i <= m;i++) {                 //在m个边里面选择
        node tmp = a[i];
        if(getf(tmp.x)!=getf(tmp.y)) {          //若两个点还未连接
            ans += tmp.z;                       //添加边权值
            cnt++;
            dis[getf(tmp.x)] = dis[getf(tmp.y)];//结合
        }
    }
    if(cnt!=n) cout<<"orz";                     //表示该图不联通
    else cout<<ans;
    return 0;
}
\end{minted}

\subsection{4.3 并查集}\label{ux5e76ux67e5ux96c6}

\begin{minted}[]{cpp}
const int MAXN = 2e5+5;
int dis[MAXN];
void init(int n) {
    for(int i = 1;i <= n;i++) dis[i] = i;
}
int getf(int pos) {
    return dis[pos]==pos?pos:dis[pos] = getf(dis[pos]);
}
void Merge(int u,int v) {
    u = getf(u), v = getf(v);
    if(u==v) return;
    dis[v] = u;
}
\end{minted}

\subsection{4.4
最近公共祖先}\label{ux6700ux8fd1ux516cux5171ux7956ux5148}

\subsubsection{4.4.1 树上倍增法}\label{ux6811ux4e0aux500dux589eux6cd5}

\begin{minted}[]{cpp}
#include<iostream>
#include<cmath>
#include<vector>
#include<queue>
#include<algorithm>
#define fast ios::sync_with_stdio(0),cin.tie(0),cout.tie(0)
#define ll long long
using namespace std;
const int MAXN = 5e5+5;
vector<int> g[MAXN];
int dep[MAXN], f[MAXN][20];
int n, m, s, siz;
void bfs(int s) {//s为根节点
    siz = (int)(log(n)/log(2)) + 1;
    queue<int> q;
    q.push(s);
    dep[s] = 1;
    while(!q.empty()) {
        int x = q.front();q.pop();
        for(int i = 0;i < g[x].size();i++) {
            int dot = g[x][i];
            if(dep[dot]) continue;
            dep[dot] = dep[x] + 1;
            f[dot][0] = x;
            for(int j = 1;j <= siz;j++)
                f[dot][j] = f[f[dot][j-1]][j-1];
            q.push(dot);
        }
    }
}
int lca(int x, int y) {
    if(dep[x] > dep[y]) swap(x, y);//让y为深度最大的点,使y向上寻找
    for(int i = siz;i >= 0;i--)
        if(dep[f[y][i]] >= dep[x]) y = f[y][i];
    if(x == y) return x;
    for(int i = siz;i >= 0;i--) 
        if(f[x][i] != f[y][i]) x = f[x][i], y = f[y][i];
    return f[x][0];//父节点即为LCA
}
int main() {
    fast;
    cin>>n>>m>>s;
    for(int i = 1;i <= n-1;i++) {
        int u, v;
        cin>>u>>v;
        g[u].push_back(v);//存边
        g[v].push_back(u);
    }
    bfs(s);//预处理
    for(int i = 1;i <= m;i++) {
        int a,b;
        cin>>a>>b;
        cout<<lca(a,b)<<"\n";
    }
    return 0;
}
\end{minted}

\subsection{4.5 二分图匹配}\label{ux4e8cux5206ux56feux5339ux914d}

\begin{minted}[]{cpp}
const int MAXN = 505;
int k,m,n;
int vis[MAXN],match[MAXN];                          //vis标记当前对象是否判断过
//match保存当前对象匹配的值(保存起点值)
vector<int> g[MAXN];                                //存图
int dfs(int u) {//起始点
    for(int i = 0;i < g[u].size();i++) {            //遍历所有指向边找匹配
        int now = g[u][i];                          //指向点
        if(vis[now]) continue;                      //跳过已判断的
        vis[now] = 1;                               //标记
        if(match[now]==0||dfs(match[now])) {        //还未匹配,或已匹配的有新的匹配对象
            match[now] = u;                         //保存匹配的值(u->now)
            return 1;
        }
    }
    return 0;//没找到,则说明不存在此增广路(匹配)
}
int hungarian() {//计算二分匹配的最大数量
    int ans = 0;
    for(int i = 1;i <= m;i++) {                     //遍历起始点集合的所有元素
        for(int j = 1;j <= n;j++) vis[j] = 0;       //标记清零
        if(dfs(i)) ans++;                           //若有找到一个匹配则增加
    }
    return ans;
}
int main() {
    while(cin>>k&&k) {
        memset(match,0,sizeof(match));              //初始化
        cin>>m>>n;
        for(int i = 1;i <= m;i++) g[i].clear();     //初始化
        for(int i = 1;i <= k;i++) {
            int x,y;
            cin>>x>>y;
            g[x].push_back(y);                      //存图
        }
        //封装后,直接调用
        cout<<hungarian()<<"\n";
    }
    return 0;
}
\end{minted}

\newpage

\pagestyle{fancy}

\fancyhead[RO,LE]{\textbf{ACM 模板  by Lese}}

\section{5 其他}\label{ux5176ux4ed6}

\subsection{5.1 宏定义}\label{ux5b8fux5b9aux4e49}

\begin{minted}[]{cpp}
// #include <bits/stdc++.h>
#include<iostream>
#include<cstring>
#include<cstdio>
#include<cmath>
#include<vector>
#include<string>
#include<sstream>
#include<queue>
#include<stack>
#include<map>
#include<set>
#include<utility>
#include<bitset>
#include<algorithm>
#define fast ios::sync_with_stdio(0),cin.tie(0),cout.tie(0)
#define ll long long
#define INF 0x3f3f3f3f
#define pii pair<int,int>
#define fi first
#define se second
using namespace std;
const int MAXN = 2e5+5;
const int MOD = 1e9+7;
\end{minted}

\subsection{5.2 高精度}\label{ux9ad8ux7cbeux5ea6}

高精度,支持加法和乘法

\begin{minted}[]{cpp}
struct BigInt {
    const static int mod = 10000;
    const static int DLEN = 4;
    const static int MAX = 700;
    int a[MAX], len;
    BigInt() {
        memset(a, 0, sizeof(a));
        len = 1;
    }
    BigInt(int v) {
        memset(a, 0, sizeof(a));
        len = 0;
        do {
            a[len++] = v%mod;
            v /= mod;
        }while(v);
    }
    BigInt(const char s[]) {
        memset(a, 0, sizeof(a));
        int L = strlen(s);
        len = L/DLEN;
        if(L%DLEN) len++;
        int index = 0;
        for(int i = L-1;i >= 0;i -= DLEN) {
            int t = 0;
            int k = i - DLEN + 1;
            if(k < 0) k = 0;
            for(int j = k;j <= i;j++) 
                t = t*10 + s[j] - '0';
            a[index++] = t;
        }
    }
    BigInt operator + (const BigInt &b) const {
        BigInt res;
        res.len = max(len, b.len);
        for(int i = 0;i <= res.len;i++) 
            res.a[i] = 0;
        for(int i = 0;i <res.len;i++) {
            res.a[i] += ((i < len)?a[i]:0)+((i < b.len)?b.a[i]:0);
            res.a[i+1] += res.a[i]/mod;
            res.a[i] %= mod;
        }
        if(res.a[res.len] > 0) res.len++;
        return res;
    }
    BigInt operator *(const BigInt &b) const {
        BigInt res;
        for(int i = 0;i < len;i++) {
            int up = 0;
            for(int j = 0;j < b.len;j++) {
                int temp = a[i]*b.a[j] + res.a[i+j] + up;
                res.a[i+j] = temp%mod;
                up = temp/mod;
            }
            if(up != 0) 
                res.a[i + b.len] = up;
        }
        res.len = len + b.len;
        while(res.a[res.len - 1] == 0 && res.len > 1) res.len--;
        return res;
    }
    void output() {
        printf("%d", a[len-1]);
        for(int i = len-2;i >= 0;i--) 
            printf("%04d", a[i]);
        printf("\n");
    }
};
int main() {
    char a[3006], b[3006];
    cin>>a>>b;
    BigInt A(a), B(b);
    // A = A+B;
    A = A*B;
    A.output();
    return 0;
}
\end{minted}

\subsection{5.3 手动扩栈}\label{ux624bux52a8ux6269ux6808}

\begin{minted}[]{cpp}
//放在主函数前面即可

int size=256<<20;//256MB
char*p=(char*)malloc(size)+size;
__asm__ __volatile__("movq %0, %%rsp\n" :: "r"(p));//64bit

int __size__ = 256<<20;
char *__p__ = (char *)malloc(__size__)+__size__;
__asm__("movl %0,%%esp\n"::"r"(__p__)); //32bit

//最后需要写一句 exit(0); 退出程序
\end{minted}

\subsection{5.4 输入输出挂}\label{ux8f93ux5165ux8f93ux51faux6302}

快读(fread版):

\begin{minted}[]{cpp}
inline char gch()
{
    static char buf[100010], *h = buf, *t = buf;
    return h == t && (t = (h = buf) + fread(buf, 1, 100000, stdin), h == t) ? EOF : *h ++;
}
template<class T>
inline void re(T &x)
{
    x = 0;
    char a; bool b = 0;
    while(!isdigit(a = gch()))
        b = a == '-';
    while(isdigit(a))
        x = (x << 1) + (x << 3) + a - '0', a = gch();
    if(b == 1)
        x *= -1;
}
\end{minted}

快写:

\begin{minted}[]{cpp}
template <typename T>
void o(T p) {
    static int stk[70], tp;
    if (p == 0) {
        putchar('0');
        return;
    }
    if (p < 0) {
        p = -p;
        putchar('-');
    }
    while (p) stk[++tp] = p % 10, p /= 10;
    while (tp) putchar(stk[tp--] + '0');
}
\end{minted}

\subsection{5.5
输入输出重定向}\label{ux8f93ux5165ux8f93ux51faux91cdux5b9aux5411}

\begin{minted}[]{cpp}
freopen("dat.in","r",stdin);
freopen("dat.out","w",stdout);
fclose(stdin);
fclose(stdout);
\end{minted}

\subsection{5.6
字符串流输入}\label{ux5b57ux7b26ux4e32ux6d41ux8f93ux5165}

\begin{minted}[]{cpp}
#include<iostream>
#include<string>
#include<sstream>
using namespace std;
int main() {
    string s,tmp;
    while(getline(cin, s)) {
        stringstream ss(s);
        while(ss>>tmp) {//速度很慢
            cout<<tmp<<"\n";
        }
    }
    return 0;
}
\end{minted}

\subsection{5.7 VIM现场赛配置}\label{vimux73b0ux573aux8d5bux914dux7f6e}

\begin{minted}[]{bash}
syntax on
set nu
set tabstop=4
set shiftwidth=4
set cin
colo evening
set mouse=a
\end{minted}

\end{document}
